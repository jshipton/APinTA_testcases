\documentclass{article}
\usepackage{natbib}
\usepackage{amsmath}

\newcommand{\Jemmacomment}[1]{\emph{\bfseries Jemma says: #1}}

\title{Advanced Parallel in Time Algorithms for Partial Differential Equations: Testcases}
\author{J. Shipton}

\begin{document}

\maketitle

This document collates and describes the test cases for the Weather
and Climate Prediction and Fusion Modelling use cases for the APinTA
PDEs Excalibur project. We present the information required to
reproduce the tests and model performance metrics. Where possible, we
will reference published material and not repeat the details here, but
will add any further information as required.

\section*{Weather and Climate Prediction}

The tests described in the following subsections comprise test suite 1
and will update those published in
\citet{williamson1992standard}. This suite consists of linear and
nonlinear shallow water tests on the plane and sphere. Linear tests
are useful for benchmarking the exponential integrators we are
developing since these require splitting the equations into linear and
nonlinear parts.

\subsection{Geostrophic balance}
The ability of a model to maintain geostrophic balance is of key
importance for weather and climate prediction. The solid body rotation
test (number 2 in \citet{williamson1992standard}) is initialised in
geostrophic balance and is commonly used to demonstrate this
property. A linearised version is described in
\citet{weller2012computational}.

\subsection{Gaussian initial condition}
Initial conditions comprised of Gaussian bumps added to a flow at rest
provide a way to analyse numerical dispersion errors. They have been
used for the linear shallow water equations on the f-plane and
f-sphere \citep{schreiber2018beyond, schreiber2019parallel}.

\subsection{Rossby waves}
The Rossby wave test case specified in \citet{williamson1992standard}
has been the subject of much discussion since the initial conditions
are not a solution of the shallow water equations and the wave becomes
unstable at long times. \Jemmacomment{add Paldor papers here}.

\subsection{Barotropic instability}
The unstable barotropic jet described in \citet{galewsky2004initial}
is a commonly used testcase for the nonlinear shallow water
equations. The instability causes the jet to roll up into vortices the
pattern of which has been reproduced by many models, although there is
no analytical solution to compare to. The complex flow presents a
challenge to numerical models and mimics the multiscale, nonlinear
behaviour found in the real atmosphere. The test has been used to
investigate the best way to treat the Coriolis term, with
\citet{schreiber2019exponential} finding that they achieve the best
wallclock-time to error when the incorporate the Coriolis term into
the nonlinear operator in their REXI computation.

\section*{Fusion Modelling}

\subsection{Anisotropic heat equation}

\subsection{Plasma model}

One of the key challenges in delivering a viable fusion reactor is understanding
the complex instabilities that occur near the plasma edge and lead to the plasma
transitioning to turbulent flow. In particular, edge turbulence allows plasma to
cross magnetic field lines and escape confinement, increasing the energy
requirements needed to achieve a self-sustaining reaction. Additionally, being
able to model and predict the location and amount of energy transfer to the
tokamak’s interior walls has huge implications for the design of fusion
reactors, as the materials lining the interior wall of the tokamak must be able
to withstand the temperatures that are generated by the transport of thermal
energy from the plasma’s edge. Understanding the mechanisms that dictate the
turbulent regime, in order both to control and to predict power deposition, is
therefore highly important in the context of long-term fusion research, in terms
of reducing the energy barrier to achieving ignition, improving the efficiency
of reactors in potential commercial usage and refining the design of the
reactor.

One such mechniams 

Dift wave turbu

The inv

A simple model for drift-wave turbulence, based on 

\begin{align*}
  \frac{\partial\zeta}{\partial t} + [\phi, \zeta] &= \alpha (\phi - n) \\
  \frac{\partial n}{\partial t} + [\phi, n] &= \alpha (\phi - n) - \kappa \frac{\partial\phi}{\partial y}
\end{align*}

where $\zeta$ is the vorticity, $n$ is the perturbed number density, $\phi$ is
the electrostatic potential, and

\[
  [a,b] = \frac{\partial a}{\partial x} \frac{\partial b}{\partial y} -
  \frac{\partial a}{\partial y} \frac{\partial b}{\partial x}
\]

is the canonical Poisson bracket operator. The vorticity and electrostatic
potential are related through the Poisson equation $\nabla^2\phi =
\zeta$. $\alpha$ is the adiabiacity operator (taken to be constant in this
solver), and $\kappa$ is the background density gradient scale length.

In the enclosed solver, the equations are solved in a similar manner. We
formulate the above equations in conservative form as

$$ \frac{\partial \mathbf{u}}{\partial t} + \nabla \cdot \mathbf{F}(\mathbf{u}) = \mathbf{G}(\mathbf{u}) $$

where $\mathbf{u} = (\zeta, n)$,
$\mathbf{F}(\mathbf{u}) = [ \mathbf{v}_E \phi, \mathbf{v}_E \zeta ]$ with the
drift velocity $\mathbf{v}_E = (\partial_y \phi, -\partial_x \phi)$, and
$\mathbf{G}$ contains the remaining source terms on the right hand side. This is
then discretised using a discontinuous Galerkin formulation; this has the
advantage of increasingly stability and means that the hyperviscosity term that
sometimes appears on the right hand side is not required.



\bibliographystyle{plainnat}
\bibliography{references}

\end{document}
