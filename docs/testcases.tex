\documentclass{article}
\usepackage{natbib}

\newcommand{\Jemmacomment}[1]{\emph{\bfseries Jemma says: #1}}

\title{Advanced Parallel in Time Algorithms for Partial Differential Equations: Testcases}
\author{J. Shipton}

\begin{document}

\maketitle

This document collates and describes the test cases for the Weather
and Climate Prediction and Fusion Modelling use cases for the APinTA
PDEs Excalibur project. We present the information required to
reproduce the tests and model performance metrics. Where possible, we
will reference published material and not repeat the details here, but
will add any further information as required.

\section*{Weather and Climate Prediction}

The tests described in the following subsections comprise test suite 1
and will update those published in
\citet{williamson1992standard}. This suite consists of linear and
nonlinear shallow water tests on the plane and sphere. Linear tests
are useful for benchmarking the exponential integrators we are
developing since these require splitting the equations into linear and
nonlinear parts.

\subsection{Geostrophic balance}
The ability of a model to maintain geostrophic balance is of key
importance for weather and climate prediction. The solid body rotation
test (number 2 in \citet{williamson1992standard}) is initialised in
geostrophic balance and is commonly used to demonstrate this
property. A linearised version is described in
\citet{weller2012computational}.

\subsection{Gaussian initial condition}
Initial conditions comprised of Gaussian bumps added to a flow at rest
provide a way to analyse numerical dispersion errors. They have been
used for the linear shallow water equations on the f-plane and
f-sphere \citep{schreiber2018beyond, schreiber2019parallel}.

\subsection{Rossby waves}
The Rossby wave test case specified in \citet{williamson1992standard}
has been the subject of much discussion since the initial conditions
are not a solution of the shallow water equations and the wave becomes
unstable at long times. \Jemmacomment{add Paldor papers here}.

\subsection{Barotropic instability}
The unstable barotropic jet described in \citet{galewsky2004initial}
is a commonly used testcase for the nonlinear shallow water
equations. The instability causes the jet to roll up into vortices the
pattern of which has been reproduced by many models, although there is
no analytical solution to compare to. The complex flow presents a
challenge to numerical models and mimics the multiscale, nonlinear
behaviour found in the real atmosphere. The test has been used to
investigate the best way to treat the Coriolis term, with
\citet{schreiber2019exponential} finding that they achieve the best
wallclock-time to error when the incorporate the Coriolis term into
the nonlinear operator in their REXI computation.

\section*{Fusion Modelling}

\subsection{Anisotropic heat equation}

\subsection{Plasma model}

\bibliographystyle{plainnat}
\bibliography{references}

\end{document}
